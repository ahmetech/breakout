\documentclass[10pt,twocolumn,letterpaper]{article}

\usepackage{cvpr}
\usepackage{times}
\usepackage{epsfig}
\usepackage{graphicx}
\usepackage{amsmath}
\usepackage{amssymb}
\usepackage{bm}

% Include other packages here, before hyperref.

% If you comment hyperref and then uncomment it, you should delete
% egpaper.aux before re-running latex.  (Or just hit 'q' on the first latex
% run, let it finish, and you should be clear).
%\usepackage[pagebackref=true,breaklinks=true,letterpaper=true,colorlinks,bookmarks=false]{hyperref}

\cvprfinalcopy % *** Uncomment this line for the final submission

\def\cvprPaperID{****} % *** Enter the CVPR Paper ID here
\def\httilde{\mbox{\tt\raisebox{-.5ex}{\symbol{126}}}}

% Pages are numbered in submission mode, and unnumbered in camera-ready
\ifcvprfinal\pagestyle{empty}\fi
\begin{document}

%%%%%%%%% TITLE
\title{W4735 Visual Interfaces to Computers\\Final Project Report}
\author{Jie Huang and Shao-Chuan Wang\\
Department of Computer Science\\
Columbia University\\
{\tt\small \{jh3105, sw2644\}@columbia.edu}
% For a paper whose authors are all at the same institution,
% omit the following lines up until the closing ``}''.
% Additional authors and addresses can be added with ``\and'',
% just like the second author.
% To save space, use either the email address or home page, not both
}

\maketitle
\thispagestyle{empty}

%%%%%%%%% ABSTRACT
\begin{abstract}
\end{abstract}

%%%%%%%%% BODY TEXT

\section{Introduction}

Visual interfaces in the applications of human-computer interface 
have been progressed tremendously in recent years. In particular, an outside-out 
visual system \cite{outout} which does not require players to wear 
any extra devices provides flexibilities in gaming applications, if 
the motion of the body can be accurately captured.
As we can see from the recent example, Microsoft Kinect, is the 
first vision-based commercial game platform, and it does 
create a lot of new paradigms for playing video games with human bodies. 

However, the idea of using gesture or human body as game
controllers appeared in 1990s. For example, 
Freeman et al. \cite{cvicg, cvfcg} implemented a computer vision based 
computer game system, based on image moments and orientation histograms, and 
use the chips to provide real-time interactive responses to the players. 

More recently, H\"{a}m\"{a}l\"{a}inen \cite{childgame} designed a perceptual user interface for
controlling a flying cartoon-animated dragon in a
physically and vocally interactive computer game for 4 to 9 years old children.
Lu et al. \cite{racecar} proposed a vision-based 3D racing car game controlling
method by analyzing the distance of two fists positions of the player
from the camera to control the direction of the racing car. Similarly, 
Schlattmann \cite{3dgames} et al. demonstrated three 3D real-time games via bare-handed 
interactions. All studies show higher usability via vision-based gesture controls 
in their usability tests.


In this proposal, we propose to to build a gesture-based game controller, which 
requires the modern computer vision technology. For a detailed review on visual 
interface of hand gestures and human motion capture, please see 
the review articles \cite{pamireview, cviureview}.


\section{Conclusion and future direction}

{\small
\bibliographystyle{ieee}
\bibliography{egbib}
}

\end{document}
